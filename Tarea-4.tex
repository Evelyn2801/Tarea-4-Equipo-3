% Options for packages loaded elsewhere
\PassOptionsToPackage{unicode}{hyperref}
\PassOptionsToPackage{hyphens}{url}
%
\documentclass[
  12pt,
]{article}
\usepackage{amsmath,amssymb}
\usepackage{lmodern}
\usepackage{iftex}
\ifPDFTeX
  \usepackage[T1]{fontenc}
  \usepackage[utf8]{inputenc}
  \usepackage{textcomp} % provide euro and other symbols
\else % if luatex or xetex
  \usepackage{unicode-math}
  \defaultfontfeatures{Scale=MatchLowercase}
  \defaultfontfeatures[\rmfamily]{Ligatures=TeX,Scale=1}
  \setmainfont[]{Times New Roman}
  \setsansfont[]{Times New Roman}
\fi
% Use upquote if available, for straight quotes in verbatim environments
\IfFileExists{upquote.sty}{\usepackage{upquote}}{}
\IfFileExists{microtype.sty}{% use microtype if available
  \usepackage[]{microtype}
  \UseMicrotypeSet[protrusion]{basicmath} % disable protrusion for tt fonts
}{}
\makeatletter
\@ifundefined{KOMAClassName}{% if non-KOMA class
  \IfFileExists{parskip.sty}{%
    \usepackage{parskip}
  }{% else
    \setlength{\parindent}{0pt}
    \setlength{\parskip}{6pt plus 2pt minus 1pt}}
}{% if KOMA class
  \KOMAoptions{parskip=half}}
\makeatother
\usepackage{xcolor}
\usepackage[margin=1in]{geometry}
\usepackage{graphicx}
\makeatletter
\def\maxwidth{\ifdim\Gin@nat@width>\linewidth\linewidth\else\Gin@nat@width\fi}
\def\maxheight{\ifdim\Gin@nat@height>\textheight\textheight\else\Gin@nat@height\fi}
\makeatother
% Scale images if necessary, so that they will not overflow the page
% margins by default, and it is still possible to overwrite the defaults
% using explicit options in \includegraphics[width, height, ...]{}
\setkeys{Gin}{width=\maxwidth,height=\maxheight,keepaspectratio}
% Set default figure placement to htbp
\makeatletter
\def\fps@figure{htbp}
\makeatother
\setlength{\emergencystretch}{3em} % prevent overfull lines
\providecommand{\tightlist}{%
  \setlength{\itemsep}{0pt}\setlength{\parskip}{0pt}}
\setcounter{secnumdepth}{-\maxdimen} % remove section numbering
\def\tablename{Tabla}
\def\figurename{Figura}
\usepackage{amsmath}
\usepackage{float}
\ifLuaTeX
  \usepackage{selnolig}  % disable illegal ligatures
\fi
\IfFileExists{bookmark.sty}{\usepackage{bookmark}}{\usepackage{hyperref}}
\IfFileExists{xurl.sty}{\usepackage{xurl}}{} % add URL line breaks if available
\urlstyle{same} % disable monospaced font for URLs
\hypersetup{
  pdftitle={TAREA},
  pdfauthor={María Isabel cabrales Soria},
  hidelinks,
  pdfcreator={LaTeX via pandoc}}

\title{TAREA}
\author{María Isabel cabrales Soria}
\date{2023-05-07}

\begin{document}
\maketitle

\textbf{3. Estudie el financiamiento del sistema bancario en México a la
luz del concepto de ``transformación de madurez''}

\textbf{(a) Obtenga, del SIE Financiamiento e información financiera de
intermediarios financieros del Banco de México, información de las
formas de financiamiento del sector bancario (comercial) mexicano, y
haga gráficas describiendo la evolución en el tiempo de las distintas
tipos de financiamiento (depósitos a la vista, financiamiento de mercado
y otros) y de la proporción que cada uno representa del total. Es decir,
hay que producir dos gráfcas de series de tiempo en la que el valor
total está constituido por varias partes intermedias.}

Se obtuvo una serie de datos provenientes del apartado ``Principales
activos y pasivos de la banca comercial (metodología 2018)'', dados por
saldos nominales en miles de pesos. Estos datos se dividen en cinco
categorías, las cuales se distinguen a continuación:

-Captación: Para que las instituciones de crédito (bancos múltiples y
bancos de desarrollo) cumplan su función de intermediación, requieren
captar recursos, tanto del sector privado no bancario residente, de la
Banca de desarrollo, de otros intermediarios financieros públicos
(Fideicomisos de fomento), sector público no residente, entre otros. En
este apartado se agregan las cuentas de cheques, depósitos de nómina y
otros retirábles sólo con tarjeta de debito, cuentas de ahorro,
depósitos a Plazo y Pagarés con Rendimiento Liquidable al Vencimiento.

-Acreedores por reporto de valores: Es una operación de crédito en
virtud de la cual el reportador, en este caso la banca de desarrollo,
Banco de México, y otras instituciones financieras (Incluye al gobierno
de la Ciudad de México, a los organismos descentralizados, empresas
productivas del Estado, empresas de participación estatal y al IPAB)
adquieren por una suma de dinero la propiedad de títulos de crédito, y
se obliga a transferir al reportado la propiedad de otros tantos títulos
de la misma especie, en el plazo convenido y contra reembolso del mismo
precio, más un premio.

-Financiamiento Interno: Importe que recibe la banca comercial en
efectivo o en especie de acreedores nacionales y que son, además, motivo
de autorización y registro por parte de la Secretaría de Hacienda y
Crédito Público, sin importar el tipo de moneda en que se documenten.

-Financiamiento Externo: Préstamos recibidos por la banca comercial de
entidades financieras nacionales , los cuales pueden ser en efectivo o
en forma de acreedores nacionales.

-Otros pasivos más capital: Préstamos recibidos por la banca comercial
de entidades financieras extranjeras, los cuales pueden ser en efectivo
o en forma de acreedores extranjeros.Son obligaciones subordinadas en
poder del sector privado residente y no residente, además de las
reservas para previsión de riesgos crediticios, entre otras cuentas por
pagar.

A continuación se muestra la evolución de estas fuentes de
financiamiento en cada periodo, comenzando en julio de 2009 hasta marzo
de 2023. Se consideró además la serie del Índice Nacional de Precios al
Productor con base julio 2019 para deflactar los valores. Al utilizar el
INPP como un índice de deflación, se eliminan los efectos de la
inflación en los valores nominales y se pueden comparar valores
económicos de diferentes períodos de tiempo de manera más precisa.

\includegraphics{Tarea-4_files/figure-latex/unnamed-chunk-7-1.pdf}

\end{document}
